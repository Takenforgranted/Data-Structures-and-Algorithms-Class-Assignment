\documentclass[UTF8]{ctexart}
\usepackage{amsmath}
\usepackage{fix-cm}  % 添加 fix-cm 宏包

\pagestyle{empty}  % 移除页码

\begin{document}

\textbf{12.证明:} 若已知一棵二叉树的先序遍历序列和中序遍历序列,则可唯一确定一棵二叉树。




\textbf{基础情况:} 对于只有一个节点的二叉树,其先序遍历和中序遍历序列都只包含一个节点,因此唯一确定了这棵二叉树。




\textbf{归纳假设:} 对于深度为 $n$ 的二叉树,已知其先序遍历序列和中序遍历序列可以唯一确定该二叉树。




\textbf{归纳步骤:} 现在考虑深度为 $n+1$ 的二叉树。首先,该树的先序遍历序列的第一个元素是根节点。然后,在中序遍历序列中找到根节点的位置,将中序遍历序列分为左子树和右子树两部分。

根据归纳假设,可以唯一确定左子树和右子树。通过递归应用相同的过程,可以唯一确定左子树和右子树的结构。因此,整个深度为 $n+1$ 的二叉树可以唯一确定。

由数学归纳法,对于所有深度为 $n$ 的二叉树,已知其先序遍历序列和中序遍历序列可以唯一确定该二叉树。证毕。

\end{document}
